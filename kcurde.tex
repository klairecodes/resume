\documentclass[11pt,letterpaper,ragged2e]{altacv}
\geometry{left=2cm,right=10cm,marginparwidth=6.8cm,marginparsep=1.2cm,top=1.25cm,bottom=1.25cm}
\ifxetexorluatex
  \setmainfont{Carlito}
\else
  \usepackage[utf8]{inputenc}
  \usepackage[T1]{fontenc}
  \usepackage[default]{lato}
\fi

\renewcommand{\itemmarker}{{\small\textbullet}}
\renewcommand{\ratingmarker}{\faCircle}

\begin{document}
\name{KLAUS CURDE}
\tagline{Software and DevOps Engineer seeking a Summer 2021 Co-op/Internship}
\personalinfo{%
  \email{\href{mailto:kcurde@gmail.com}{kcurde@gmail.com}}
  \phone{\href{tel:724-718-7281}{(724) 718-7281}}
  \github{\href{https://github.com/klauscurde}{github.com/klauscurde}}
}

%% Make the header extend all the way to the right
\begin{fullwidth}
\makecvheader
\end{fullwidth}

%% Make fonts of itemize environments slightly smaller
\AtBeginEnvironment{itemize}{\small}

%% Provide the file name containing the sidebar contents as an optional parameter to \cvsection.

\cvsection[sidebar]{EXPERIENCE}

\cvproject{System Administrator}{Computer Science House}
{
\begin{itemize}
    \item Manage Linux systems, websites, and web services in a server room for the RIT organization CSH.
    \item Exposure to enterprise-level tech, including RHEL, OKD 4, Kubernetes, Docker, Proxmox, FreeIPA, and Datadog.
    \item Offer guidance to other members looking to learn devops skills or spin up their own projects.
\end{itemize}
}
{Oct 2020 - Present} \

\cvsection{Projects}

\cvproject{ritlinks}{\href{https://github.com/klauscurde/ritlinks}{github.com/klauscurde/ritlinks}}
{
\begin{itemize}
    \item Used the Python web framework Django to create a dynamically generated website that lists a collection of all the websites RIT makes available to students as well as their descriptions.
    \item Helps new students understand which services do what, and has administrator accounts to manage adding and removing content as relevant.
    \item Runs in a Docker container.
\end{itemize}
}
{Aug 2020 - Present}


\cvproject{HuntedRC Car}{\href{https://github.com/klauscurde/Hunted_RC_Car}{github.com/Hunted\_RC\_Car}}
{
\begin{itemize}
    \item Designed the electronic internals of a remote control car that can be controlled from up to 1 kilometer away.
    \item Uses an NRF24L01 wireless module with a self-designed wireless protocol, an L298N H-bridge motor controller, and \\ Arduino Mega and Nano microcontroller boards.
    \item Coded in C++ and uses a custom designed controller. Internals are a prototype for modifying a Barbie Jeep to be wirelessly controlled as a safe, moving hunting target.
\end{itemize}
}
{May 2018 - July 2020}


\cvproject{LifeSym}{\href{https://github.com/klauscurde/LifeSim}{github.com/klauscurde/LifeSim}}{
\begin{itemize}
    \item Programmed a Java application that simulates the human body to provide health recommendations.
    \item Utilizes Metabolic Equivalent (MET) data values sourced from an academic paper to calculate an estimate of the amount of calories consumed during specific activities.
\end{itemize}
}
{Jan 2020 - July 2020}

\cvproject{CookC}{\href{https://github.com/klauscurde/CookC}{github.com/klauscurde/CookC}}
{
\begin{itemize}
    \item Wrote a small game in Python that exists entirely as an array displayed in a character LCD.
    \item Runs on a Raspberry Pi Zero W that is connected to the \\ internal power of a ThinkPad X230 laptop.
\end{itemize}
}
{Apr 2020} \

\clearpage

\nocite{*}

\end{document}
